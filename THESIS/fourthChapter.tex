\chapter{Conclusioni e possibili evoluzioni}

\section{Utilizzo di processori Dual Core}
I processi coinvolti per la replicazione dei dati e dei metadati sono due: 
\begin{itemize}
\item 
il database Postgres;
\item
il \textit{frontend nodejs} che scrive sul database.\\
\end{itemize}

Usando un processo a \textit{single-core}, i due processi si contengono il processore e si constringono a svariati \textit{context-switch} di memoria.\\
Il \textit{context-switch} \`{e} fonte di \textit{overhead} non indifferente, generato ad esempio:
\begin{itemize}
\item 
tempo impiegato a salvare in memoria lo stato;
\item
blocco e riattivazione delle \textit{pipeline} di calcolo del processore;
\item 
svuotamento e ripopolamento delle cache.
\end{itemize}

\`{E} di conseguenza necessario evitare di effettuare un numero eccessivo di \textit{context-switch}.\\

Poich\`{e} i prezzi si stanno riducendo, una possibile evoluzione \`{e} quella di usare un \textit{dual-core}, in modo che ogni processo abbia un \textit{core} dedicato e non siano eseguiti numerosi \textit{context-switch}.\\
Ci\`{o} potrebbe consentire un vantaggio nelle prestazioni

\textbf{per i processori dual-core, inserisci una piccola valutazione
}

\section{Utilizzo di dischi SSD}
Per gli SSD, viste le prestazioni attuali invece, non ci attendiamo miglioramenti, se non in termini di minori consumi.
