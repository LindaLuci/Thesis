\chapter{Conclusioni e possibili evoluzioni}
%\setcounter{section}{-1}

\item
\section{Evoluzioni hardware}

\item 
\subsection{Utilizzo di processori dual core}
I processi coinvolti per la replicazione dei dati e dei metadati sono due: 
\begin{enumerate}
\item 
il database Postgres;
\item
il \textit{frontend nodejs} che scrive sul database.
\end{enumerate}

Usando un processo a \textit{single-core}, i due processi si contendono il processore e si constringono a svariati \textit{context-switch} di memoria.\\
Il \textit{context-switch} \`{e} fonte di \textit{overhead} non indifferente, generato da:

\begin{itemize}
\item 
tempo impiegato a salvare in memoria lo stato;
\item
blocco e riattivazione delle \textit{pipeline} di calcolo del processore;
\item 
svuotamento e ripopolamento delle cache.
\end{itemize}

\`{E} di conseguenza consigliabile evitare di effettuare un numero eccessivo di \textit{context-switch}.\\

Poich\`{e} i prezzi si stanno riducendo, una possibile evoluzione \`{e} quella di usare un \textit{dual-core}, in modo che ogni processo abbia un \textit{core} dedicato e non siano eseguiti numerosi \textit{context-switch}. Questo tipo di architettura consente di aumentare la potenza di calcolo di una CPU senza aumentare la frequenza di lavoro. \\
Ci\`{o} potrebbe consentire numerosi vantaggi, tra cui miglioramenti per quanto riguardano le prestazioni.

\item
\subsection{Utilizzo di dischi SSD}
Per gli SSD (\textit{solid-state disk}), viste le prestazioni attuali, non sono attesi eccessivi miglioramenti, se non in termini di minori consumi.\\

Esaminiamo i principali fattori di vantaggio dell'utilizzo di dischi SSD.\\
Quando il costo di \textit{storage}, ossia il costo a \verb"GB", sar\`{a} lo stesso degli \textit{hard disk}, l'utilizzo degli SSD consentir{a} di diminuire il tasso di guasto.\\ 
L'unit\`{a} a stato solido \`{e} di fatto pi\`{u} resistente, dal momento che non contengono parti in movimento (come il motore e il disco magnetico degli HD).\\
Non ci sar\`{a} alcuna perdit\`{a} del disco, di conseguenza sar\`{a} possibile ridurre le ridondanze di dati. L'utilizzo della parit\`{a}, ad esempio, non sar\`{a} pi\`{u} conveniente, tuttavia diverr\`{a} fonte di eccessiva occupazione del disco. \\

Il costo degli SSD incide sul costo di servizio. \\
I costi di servizio sono quelli che sono pagati dai clienti per avere uno  \textit{storage} dei loro dati.\\
Sarebbe possibile diminuire i costi se ci fosse la possibili\`{a} di usare meno \textit{storage} per storare ogni dato degli utenti. \\
Per mantenere nello \textit{storage} \verb"1 GB", sono occupati in realt\`{a} \verb"2,3 GB", in modo da garantire la sicurezza del dato in caso di perdita. \\
Con l'utilizzo degli SSD sarebbe possibile occupare solo \verb"2 GB" per ogni \verb"GB" del cliente, concendendo un \verb"30%" di spazio in meno che per\`{o} ci permette di diminuire il costo a \verb"GB".\\

L'utilizzo di questo tipo di harware influisce sui costi di esercizio, ovvero il valore che \`{e} spesa dall'azienda anche qualora non ci sia alcun cliente che usufruisce dello spazio, come il consumo elettrico necessario.\\
Il costo di esercizio si pu\`{o} diminuire riducendo i consumi. Se l'uso degli SSD permettesse un calo dei consumi, da un punto di vista di costi, sarebbe un ulteriore vantaggio. Ci sarebbe un garantito taglio dei costi.\\

Tuttavia attualmente il costo degli SSD \`{e} estremamente alto e, per questo, non \`{e} conveniente.

\item
\section{Evoluzioni software}
In questa tesi \`{e} stato esaminato come la replicazione dei dati possa essere gestito da Postgres e dai cluster di database.\\
Una possibile evoluzione software \`{e} di non limitarsi pi\`{u} a offrire un servizio RESTful API basato su \textit{object storage}, ma in qualcosa di pi\`{u} evoluto. Nello specifico, potrebbe essere un software che contenga un linguaggio di query, permettendo di aggregare i dati in un altro modo. Cominciare cos\`{i} a dare degli strumenti di analisi sui dati direttamente sullo \textit{storage} invece di dare soltanto la possibilit\`{a} di accedere ai dati per caricarli o scaricarli dallo spazio virtuale.\\
Questo tipo di servizio \`{e} sempre pi\`{u} richiesto.
%cioè storage un dato ridammi il dato

%(\textit{REpresentational State Transfer}, ovvero una rappresentazione del trasferimento di stato di un determinato dato).


% API si intende Application Programming Interface e non è altro che la creazione di una serie di endpoint (delle URL) che rispondono alle richieste fatte da uno sviluppatore.