\chapter{Definizione del progetto}
\setcounter{section}{1}
\item
\subsection{Architettura del progetto}
\textbf{concept Provider-Subscriber - da scegliere dove inserirlo}
Master to multiple cascaded slaves
The basic structure of the systems combined in a Slony-I installation is a
master with one or more slaves nodes. Not all slave nodes must receive the replication
data directly from the master. Every node that receives the data from a
valid source can be configured to be able to forward that data to other nodes.
There are three distinct ideas behind this capability. The first is scalability.
One database, especially the master that receives all the update transactions
from the client applications, has only a limited capability to satisfy the slave
nodes queries during the replication process. In order to satisfy the need for a big
number of read-only slave systems it must be possible to cascade.
The second idea is to limit the required networ k bandwidth for a backup site
while keeping the ability to have multiple slaves at the remote location.
The third idea is to be able to configure failover scenar ios. In a master to
multiple slave configuration, it is unlikely that all slave nodes are exactly in the
same synchronization status when the master fails. To ensure that one slave can
be promoted to the master it is necessary that all remaining systems can agree
on the status of the data. Since a committed transaction cannot be rolled back,
this status is undoubtly the most recent sync status of all remaining slave nodes.
The delta between this one and every other node must be easily and fast generated
and applied at least to the new master (if that’s not the same system) before
the promotion can occur.
Nodes, Sets and forwarding
The Slony-I replication system can replicate tables and sequence numbers.
Replicating sequence numbers is not unproblematic and is discussed in more
detail in section 2.3.
Table and sequence objects are logically grouped into sets. Every set
should contain a group of objects that is independant from other objects originating
from the same master. In shor t, all tables that have relationships that could
be expressed as foreign key constraints and all the sequences used to generate
any ser ial numbers in these tables should be contained in one and the same set.
Figure 1 illustrates a replication configuration that has 2 data sets with different
origins. To replicate both date sets to NodeC it is not required that Node C
really communicates with the origin of Set 1. This scenario has full redundancy
for every node. Obviously if Node C fails, the masters of Set 1 and Set2 are still
alive, no problem. If Node A fails, Node B can get promoted to the master of both
sets. The tricky situation is if Node B fails.
In the case Node B fails, Node C needs to get promoted to the master of
Set 2 and it must continue replicating Set 1 from Node A. For that to be possible,
Node A must have knowledge about Node C and its subscription to Set 1. Generally
speaking, every node that stores replication log infor mation must keep it until
all subscribers of the affected set are known to have replicated that data.
To simplify the logic, the configuration of the whole networ k with all nodes,
sets and subscriptions will be forwarded to and stored on all nodes. Because the
sets,anode is not subscribed to must not even exist in its database, this does
not include the infor mation about what tables and sequences are included in any
specific set.
\subsection{Simulazione di un filesystem distribuito (Dati e Metadati)}
\section{Considerazioni statistiche sulla ridondanza sul dato}
\section{Considerazioni statistiche sulla ridondanza del metadato}
\section{Resilienza ai cambiamenti di rete}