\chapter{Definizione del quadro sperimentale}
\textbf{DA RIVEDERE - http://www.slony.info/images/Slony-I-concept.pdf}\\
Failover: While it is relatively easy to tell in a master to multiple slave scenar io
which of the slaves is most recent at the time the master fails, it is
near ly impossible to tell the actual row delta between two slaves. So in
the case of a failing master, one slave can be promoted to the master,
but all other slaves need to be re-synchronized with the new master.
Perfor mance:
Stor ing the logging infor mation in one or ver y fe w rotating log tables
means that the replication engine can retrieve the actual data for one
replication step with ver y fe w quer ies that select from one table only. In
contrast to that a system that fetches the current values from the application
tables at replication time needs to issue the same number of
quer ies per replicated table and these queries will be joining the log
table(s) with the application data table. It is obvious that this systems
perfor mance will be reverse proportional to the number of replicated
tables. At some time the complete delta to be applied, which can not
be split as pointed out already, will cause the PostgreSQL database
system to require less optimal than in memory hash join query plans to
deal with the number of rows returned by these queries and the replication
system will be unable to ever catch up unless the wor kload on
the master drops significantly
\section{Lancio in configurazione 1}
\section{Lancio in configurazione 2}
\section{Lancio in configurazione 3}